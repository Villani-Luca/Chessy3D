\documentclass[conference]{IEEEtran}
\IEEEoverridecommandlockouts
% The preceding line is only needed to identify funding in the first footnote. If that is unneeded, please comment it out.
%Template version as of 6/27/2024

\usepackage{cite}
\usepackage{amsmath,amssymb,amsfonts}
\usepackage{algorithmic}
\usepackage{graphicx}
\usepackage{textcomp}
\usepackage{xcolor}
\def\BibTeX{{\rm B\kern-.05em{\sc i\kern-.025em b}\kern-.08em
    T\kern-.1667em\lower.7ex\hbox{E}\kern-.125emX}}
\begin{document}

\title{Chessy3D: A study of 3D Chessboard Detection and Pose Estimation\\
\thanks{Identify applicable funding agency here. If none, delete this.}
}

\author{\IEEEauthorblockN{Luca Villani}
\IEEEauthorblockA{
\textit{matr. 200412}\\
\textit{University of Modena and Reggio Emilia}\\
269419@studenti.unimore.it}
\and
\IEEEauthorblockN{Alessandro Mezzogori}
\IEEEauthorblockA{\textit{matr. } \\
\textit{University of Modena and Reggio Emilia}\\
271617@studenti.unimore.it}
\and
\IEEEauthorblockN{Davide Della Casa Venturelli}
\IEEEauthorblockA{\textit{matr. } \\
\textit{University of Modena and Reggio Emilia}\\
xxxxx@studenti.unimore.it}
}

\maketitle

\begin{abstract}
In this paper, we present a novel framework for chessboard corners detection
 and chess-pieces object detection. 
Our approach tackles the significant challenges posed by the variability
 of the photos taken of the chessboard and sizes, color and differences of the chess pieces.
We propose a new method for chessboard keypoint detection that utilizes a combination of
 deep learning and geometric constraints to accurately identify the keypoints on the chessboard
 and the squares within it.
This method is robust to variations in lighting, angle, and chessboard designs,
making it suitable for real-world applications.
Additionally, we developed a chess-piece object detection model that can accurately detect and classify chess pieces in various conditions.
Our study delves into the intricacies of chessboard detection, addressing the challenges of different chessboard designs and piece variations.
The proposed approach using YOLOv8 for chess-piece detection
 demonstrates superior performance in terms of accuracy and robustness compared to existing methods.
We used two datasets for training and testing our models: ChessRed2K and another small dataset found on Roboflow.
After the object detection we will propose a FEN annotation method for converting the chessboard state into a
 FEN (Forsyth-Edwards Notation) string, which is a standard notation for describing the state of a chess game.
Additionally we show a method for retrieving similar images using one hot encoding
 and a similarity search algorithm.
\end{abstract}

\section{Introduction}
The detection and analysis of keypoints in images have
become crucial tasks in the field of computer vision, with
applications ranging from object recognition and tracking
to augmented reality and autonomous driving.

Identifying
keypoints on vehicles from different perspectives poses sig-
nificant challenges. Traditional methods often rely on ex-
tensive manual annotation, which is both time-consuming
and prone to errors. Moreover, models trained on synthetic
data frequently suffer from domain-shift when applied to
real-world images, resulting in decreased performance.
This study aims to address these challenges by develop-
ing a comprehensive keypoint detection framework inspired
by FastTrakAI [1], thus focusing on enhancing automotive
safety and efficiency through innovative AI solutions. How-
ever, the inherent variability in the shapes, sizes, and ap-
pearances of these vehicles presents significant challenges
for developing robust keypoint detection models. Our ap-
proach involves identifying a set of 3D models representing
a category of vehicles and defining a set of keypoints com-
mon to all vehicles. This step ensures that our approach is
adaptable across various vehicle types. To build our dataset,
we rendered over 2000 images of these vehicles from dif-
ferent perspectives and enriched them with keypoint anno-
tations. To avoid the laborious task of manually annotating
each image, we devised a strategy that automates a signifi-
cant portion of this process. After creating the dataset, we
trained a keypoint detection model tailored to these anno-
tated images. We then rigorously tested the model on cap-
tures of a 3D model never seen during the training phase.
This testing approach evaluates the model’s robustness and
its ability to generalize to new, unseen data. By following
this methodology, we aimed to develop a keypoint detection
system that is both accurate and robust, capable of handling
the variability inherent in vehicle categories. The insights
gained from this research can contribute to the advancement
of computer vision technologies in the automotive industry
and beyond, fulfilling the objectives of FastTrakAI by DAT
and Prometeia to push the boundaries of AI-driven automo-
tive innovation..

\section{Chessboard Corners Detection}

\section{Chessboard Squares Detection}

\section{Chess-Pieces Object Detection}

\section{FEN Annotation Method}

\section{Image Similarity Search}
\section{Conclusion}



\begin{thebibliography}{00}
\bibitem{b1} G. Eason, B. Noble, and I. N. Sneddon, ``On certain integrals of Lipschitz-Hankel type involving products of Bessel functions,'' Phil. Trans. Roy. Soc. London, vol. A247, pp. 529--551, April 1955.
\bibitem{b2} J. Clerk Maxwell, A Treatise on Electricity and Magnetism, 3rd ed., vol. 2. Oxford: Clarendon, 1892, pp.68--73.
\bibitem{b3} I. S. Jacobs and C. P. Bean, ``Fine particles, thin films and exchange anisotropy,'' in Magnetism, vol. III, G. T. Rado and H. Suhl, Eds. New York: Academic, 1963, pp. 271--350.
\bibitem{b4} K. Elissa, ``Title of paper if known,'' unpublished.
\bibitem{b5} R. Nicole, ``Title of paper with only first word capitalized,'' J. Name Stand. Abbrev., in press.
\bibitem{b6} Y. Yorozu, M. Hirano, K. Oka, and Y. Tagawa, ``Electron spectroscopy studies on magneto-optical media and plastic substrate interface,'' IEEE Transl. J. Magn. Japan, vol. 2, pp. 740--741, August 1987 [Digests 9th Annual Conf. Magnetics Japan, p. 301, 1982].
\bibitem{b7} M. Young, The Technical Writer's Handbook. Mill Valley, CA: University Science, 1989.
\bibitem{b8} D. P. Kingma and M. Welling, ``Auto-encoding variational Bayes,'' 2013, arXiv:1312.6114. [Online]. Available: https://arxiv.org/abs/1312.6114
\bibitem{b9} S. Liu, ``Wi-Fi Energy Detection Testbed (12MTC),'' 2023, gitHub repository. [Online]. Available: https://github.com/liustone99/Wi-Fi-Energy-Detection-Testbed-12MTC
\bibitem{b10} ``Treatment episode data set: discharges (TEDS-D): concatenated, 2006 to 2009.'' U.S. Department of Health and Human Services, Substance Abuse and Mental Health Services Administration, Office of Applied Studies, August, 2013, DOI:10.3886/ICPSR30122.v2
\bibitem{b11} K. Eves and J. Valasek, ``Adaptive control for singularly perturbed systems examples,'' Code Ocean, Aug. 2023. [Online]. Available: https://codeocean.com/capsule/4989235/tree
\end{thebibliography}

\vspace{12pt}
\end{document}
