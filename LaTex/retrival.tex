%% - fen is not good for retrieval tasks ( variable ecc.. )
%% - approaches of other papers ( Inverted document index paper ) small mention
%% - general approach 
%%      - from position to embedding
%%      - embedding search
%% - naive approach chose
%% - possible other approaches ( if not long enough )

\subsection{Current storage approaches}
Although fen is a perfect medium for sharing, storing position and retrieving the exact match, 
due to its definition it makes comparison between similar but not equal positions using the format
difficult without parsing and loading both positions in memory, 
becoming an hindrance especially when browsing chess databases to find games with similar position due to the computational cost of the operation.
\newline
The state of the art for storing and retrieving positions is varied based on the system that utilizes it. Most famously:
\begin{itemize}
    \item 
    { 
        For database used for human study, SCID Shane's Chess Information Database \cite{retrieval:scid} one of the most famous tools used to find, navigate and study games, powerful but does not handle search by similarity.
    }
    \item 
    {
        For storing high volume of data Lichess, one of the main chess platforms, has a developed a highly efficient format \cite{retrieval:lichess:format} that allows search alike SCID.
    }
\end{itemize}
Even if highly efficient and perfect for their respective use cases, this formats do not suit the objective set for this project being the digitalization
of physical chess position and search of similar position. 
\newline
The inability to allow search by similarity is mainly because of the subjective definition of similarity in games, as it could be approached from different points of view.
\begin{itemize}
    \item 
    {
        \textbf{Pure Positional similarity}: position of pieces on the board
    }
    \item 
    {
        \textbf{Threat similarity}: attacking and defending pieces
    }
    \item
    {
        \textbf{Structural similarity}: in most games recurring piece structures are created
    }
\end{itemize}
A novel approach that tries to merge most of the aspects related to similarity was explored by 
Ganguly, Debasis and Leveling, Johannes and Jones, Gareth 2014 \cite{retrieval:soa:ids} 
that offers a novel approach by using information retrieval methods applied to a position.
By modeling a text representation of the position utilizing positional information, square control, attacking and defending pieces 
to be used inside the standard inverted organization typical of IR. 
\newline
In this project we utilized only positional information to build a pure piece positional index similarity, sacrificing similarity accuracy
for improved performance.
\subsection{Our method}
We opted for a pure positional information approach, although using only the position of the pieces may 
return only apparently similar games with them being different,because of slight changes in position 
could lead to big difference in play, the approach is valid to explore the problem 
and given a sufficiently large database it will return games in which the position or a position less than two move difference. 
\newline
Given any position, it is encoded in a naive binary embedding of 64 slot of length 12 bits where the first 6 bits represent the white pieces 
and the other 6 bits the black pieces.
\begin{figure}[ht]
\centering
\includegraphics[width=0.75\linewidth]{retrieval/piece_to_embedding.png}
\caption{Knight to embedding}
\label{fig:retrieval:piecetoembedding}
\end{figure}
\newline
This results in a sparse binary vector embedding of length 96 bytes that is an exact one to one of the whole 
piece positional information of the given position.

\begin{algorithm}[H]
\caption{Embedding algorithm}\label{retrieval:embedding:algorithm}
\begin{algorithmic}
\State $v \gets \text{(96B array)}$
\State $\mathcal P \gets \text{Board position}$
\ForEach {$s \in \mathcal P $}
\If{$s$ is not empty}
    \State $i \gets \text{computeSquareIndex(s)}$
    \State $v[i] \gets 1$
\EndIf
\EndFor
\end{algorithmic}
\end{algorithm}

\subsection{Search database creation}
%% lumbras => pgn format (breve) % => approccio di costruzione del database => limitazione hardware 

As our chess database of choice we choose LumbrasGigabase \cite{retrieval:lumbrasgigabase}, containing 15 Million games from various sources counting
to approximately 1 billion unique game positions reached, stored in a standard format called PGN ( portable game notation ) \cite{chess:pgn} supplying 
game moves and metadata relative to each game useful to populate our database.
\newline

\begin{mdframed}
\string[Event "F/S Return Match"\string]\newline
\string[Site "Belgrade, Serbia JUG"\string]\newline
\string[Date "1992.11.04"]\string\newline
\string[Round "29"\string]\newline
\string[White "Fischer, Robert J."\string]\newline
\string[Black "Spassky, Boris V."\string]\newline
\string[Result "1/2-1/2"\string]\newline
\newline
1.e4 e5 2.Nf3 Nc6 3.Bb5 {This opening is called the Ruy Lopez.} 3...a6
4.Ba4 Nf6 5.O-O Be7 6.Re1 b5 7.Bb3 d6 8.c3 O-O 9.h3 Nb8 10.d4 Nbd7
11.c4 c6 12.cxb5 axb5 13.Nc3 Bb7 14.Bg5 b4 15.Nb1 h6 16.Bh4 c5 17.dxe5
Nxe4 18.Bxe7 Qxe7 19.exd6 Qf6 20.Nbd2 Nxd6 21.Nc4 Nxc4 22.Bxc4 Nb6
23.Ne5 Rae8 24.Bxf7+ Rxf7 25.Nxf7 Rxe1+ 26.Qxe1 Kxf7 27.Qe3 Qg5 28.Qxg5
hxg5 29.b3 Ke6 30.a3 Kd6 31.axb4 cxb4 32.Ra5 Nd5 33.f3 Bc8 34.Kf2 Bf5
35.Ra7 g6 36.Ra6+ Kc5 37.Ke1 Nf4 38.g3 Nxh3 39.Kd2 Kb5 40.Rd6 Kc5 41.Ra6
Nf2 42.g4 Bd3 43.Re6 1/2-1/2
\end{mdframed}
\begin{center}
Bobby Fischer vs Boris Spassky, 1992, Yugoslavia, 29th
\end{center}
By iterating over each game we extract and process each position by creating its embedding and Zobrist hash \cite{retrieval:zobristhash}, de factor standard 
hash used to encode a chess position in 64 bit, to efficiently store the same position used in different games. 
\newline
Although we had access to a large quantity of high quality games due to hardware memory and computation limitations 
we had to scale back the indexing of the positions to allow only 5 million instances.

\subsection{Distance function}
%% index => hamming distance
The search and retrieval of the embeddings is efficiently done trough an hamming distance HNSW index \cite{retrieval:hnsw},
a fully graph based index without any need for additional search structures, extension of earlier work on navigable small world graphs. 
\newline
The cited index enables an efficient ranking of the saved positions based on the number of flipped bits compared to the anchor embedding.
\newline
While not applicable universally we observed that a unit distance of 2 corresponds to a single different piece position, 
this could be intuitively understood by seeing it as toggling squares, one off square and one on square. 
\newline %% show image %%
This does not generalize in case of captures, castling, en passant, or any move that involve more than one piece.
\pagebreak
Through the use of the HNSW index we are able to have a considerable speedup in search and retrieval speed on 10 million indexed positions.
\begin{table}[ht]
\centering
\caption{Speed comparison with and without vector index.}
\label{retrieval:table:vectorindex}
\begin{tabular}{lc}
\textbf{Run} & \textbf{Average time} \\
HNSW & $\sim$121ms \\
None & $\sim$959ms 
\end{tabular}
\end{table}
By creating the binary embedding, chess database and vector index we have efficiently implemented a retrieval system for similar games able to return
the closest game in terms of number of moves and piece on the board to a given position.

%% SOA METHODS https://arxiv.org/html/2310.04086v3 BIBTEX: https://arxiv.org/html/2310.04086v3/#bib.bibx15
